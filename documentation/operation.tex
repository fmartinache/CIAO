% Created 2018-02-21 mer. 17:06
\documentclass[11pt]{article}
\usepackage[utf8]{inputenc}
\usepackage[T1]{fontenc}
\usepackage{fixltx2e}
\usepackage{graphicx}
\usepackage{longtable}
\usepackage{float}
\usepackage{wrapfig}
\usepackage{soul}
\usepackage{textcomp}
\usepackage{marvosym}
\usepackage{wasysym}
\usepackage{latexsym}
\usepackage{amssymb}
\usepackage{hyperref}
\tolerance=1000
\providecommand{\alert}[1]{\textbf{#1}}

\title{Mise en route de CIAO}
\author{Frantz Martinache}
\date{\today}
\hypersetup{
  pdfkeywords={},
  pdfsubject={},
  pdfcreator={Emacs Org-mode version 7.9.3f}}

\begin{document}

\maketitle

\setcounter{tocdepth}{3}
\tableofcontents
\vspace*{1cm}

\section{Après un reboot de l'ordinateur}
\label{sec-1}


Normalement, la machine devrait avoir démarré sous Linux. Si ça n'est
pas le cas, je me demande bien comment vous faites pour lire ces
quelques lignes de toute façon.

Après un reboot, il faut mettre en route les drivers de la carte de
controle de la caméra iXon et du DM ALPAO. La procédure pourrait etre
automatisée mais où serait le fun là dedans? Pour impressionner les
touristes, il faut lancer des commandes dans un terminal, qui
affichent plein de messages compliqués.
\subsection{To start the driver for the camera:}
\label{sec-1-1}



\begin{verbatim}
cd /home/ciaodev/Documents/CIAO/Andor_SDK/V2.99.30002.0/andor/script/
sudo bash ./andordrvlx_load
\end{verbatim}

Rien de très impressionnant ici: la commande renvoit un simple:

``andor driver installed''
\subsection{To start the DM driver again:}
\label{sec-1-2}



\begin{verbatim}
cd /home/ciaodev/src/driver_DM/
sudo bash ./install.sh
cd
\end{verbatim}

Ici le retour est beaucoup plus verbose: pas mal de warnings mais à
la fin, on voit:

Load kernel modules
Create configuration file gpg2x72c.conf
====================================================================
 Ref.ID | Model                          | RSW1 |  Type | Device No.  
\hrule
      1 | PEX-292144                     |    0 |    DI |    17
\hrule
      2 | PEX-292144                     |    0 |    DO |    18
====================================================================

Tout est bon. Les drivers sont chargés et on peut commencer à
intéragir avec le matériel.
\section{Lancement des serveurs de controle du matos}
\label{sec-2}


Après les drivers, il faut lancer les serveurs pour le DM et la
caméra. Le plus simple, c'est de mettre le DM en route en premier.
\subsection{Miroir déformable}
\label{sec-2-1}


La commande qui lance le serveur du DM est:


\begin{verbatim}
alpao_server &
\end{verbatim}

Le serveur du DM est pour le moment discret et n'écrit pas grand
chose après ses messages initiaux:

Use x86$_{\mathrm{64}}$ libraries.
/tmp/dmdisp0.im.shm will be created or overwritten
/tmp/dmdisp1.im.shm will be created or overwritten
/tmp/dmdisp2.im.shm will be created or overwritten
/tmp/dmdisp3.im.shm will be created or overwritten
/tmp/dmdisp4.im.shm will be created or overwritten
/tmp/dmdisp5.im.shm will be created or overwritten
/tmp/dmdisp6.im.shm will be created or overwritten
/tmp/dmdisp7.im.shm will be created or overwritten
/tmp/ciao$_{\mathrm{dm}}$$_{\mathrm{disp}}$.im.shm will be created or overwritten

On peut visualiser les 8 channels du DM simultanément avec un petit
programme:


\begin{verbatim}
dmview &
\end{verbatim}

qui devrait afficher une fenetre simple avec 2x5 panels dont seuls
les \#0 et \#8 affichent quelquechose. Ce qui est appliqué sur le \#0
c'est le flat qui a été établi après réception du DM devant
l'interféromètre de Zygo.
\subsection{la caméra iXon}
\label{sec-2-2}


Pour lancer le serveur de controle de la caméra la commande est:


\begin{verbatim}
ixon_server
\end{verbatim}

Si tout se passe bien, on entend assez vite le bruit du shutter de la
caméra. Mais il va falloir attendre un peu. En attendant, je vous
raconte une histoire:

Ici, j'évite le \& à la fin de la commande: le serveur affiche
quelquefois des choses (lorsqu'on change le temps d'intégration par
exemple): une fois que le serveur est lancé, le terminal qui l'a
lancé est condamné à rester ouvert. Une solution c'est de lancer la
meme commande depuis un terminal SCREEN (qui est installé sur la
machine). L'intéret de ce terminal est qu'il peut etre détaché mais
il vaut mieux etre familier avec ces raccourcis bizarres comme: Ctrl-A
Ctrl-D, \ldots{} donc ici, j'évite.

Le serveur de la caméra commence par mettre le détecteur à
température. La consigne pour le moment est -5 degrés Celcius. Ca
peut prendre plusieurs minutes. Mais à un moment, il va dire:

``iXon sever ready.''

La fete peut donc enfin commencer!
\section{Lancement du logiciel de controle de l'AO}
\label{sec-3}


Dans un nouveau terminal, ou un nouvel onglet du terminal en cours
(créé en appuyant sur Ctrl-Shift T, ou alors avec un clic droit sur la
fenetre du terminal), il faut lancer la commande suivante:


\begin{verbatim}
ciao_control /tmp/ixon.im.shm
\end{verbatim}

qui lance le GUI de controle de l'AO. Comme la caméra n'a encore pas
fait d'acquisition, tout est à zero, nan, vide\ldots{} mais il suffit
d'appuyer sur le bouton ``STREAM'' dans le panel gauche du GUI pour que
la situation change!

Le GUI est plus ou moins organisé en 3 panels:

\begin{itemize}
\item à gauche, c'est le controle de la caméra
\item au milieu, c'est la paramétrisation de l'ASO
\item à droite, c'est le controle de divers asservissements
\end{itemize}
\subsection{controle de la caméra}
\label{sec-3-1}


Tout se passe dans le panel gauche.
\subsection{paramétrisation de l'utilisation du WFS}
\label{sec-3-2}
\subsection{controle de front d'onde}
\label{sec-3-3}
\section{Arret de la manipe}
\label{sec-4}
\subsection{Serveur de la camera}
\label{sec-4-1}


Il faut lui envoyer la commande ``quit''. Plusieurs façons de faire:

\begin{enumerate}
\item en ligne de commande, il faut écrire:
\end{enumerate}

>> echo ``quit'' > /home/ciaodev/bin/ixon$_{\mathrm{fifo}}$$_{\mathrm{in}}$

\begin{enumerate}
\item utiliser un petit outil graphique:
\end{enumerate}

depuis un terminal:

>> ixon$_{\mathrm{ctrl}}$

Et appuyer sur le bouton ``quit''.

Dans tous les cas, on entend le shutter de la caméra qui se baisse et
le refroidissement qui s'arrete. Tout est plus silencieux.
\subsection{Serveur du DM}
\label{sec-4-2}


Pour le moment, seulement une possibilité en ligne de commande:

>> touch /home/ciaodev/bin/dm$_{\mathrm{stop}}$
\subsection{shutdown}
\label{sec-4-3}

\begin{itemize}
\item Débrancher du secteur l'alim du refroidissement de la camera
\item éteindre le boitier d'alim du DM (inter à l'arrière du boitier)
\item shutdown de l'ordinateur
\end{itemize}

\end{document}
